\documentclass[12pt]{report}

\usepackage[a4paper, margin=1.5cm]{geometry} % imposta le dimensioni della pagina

\usepackage{hyperref} % permette agli hypertext di essere cliccabili
\hypersetup{
	pdftitle={Progetto TecWeb},
	pdfauthor={Roberto Ingenito},
	colorlinks,
	linkcolor=black,
	citecolor=blue,
	filecolor=blue,
	urlcolor=blue
}


\linespread{1.2} % imposta lo spazio tra le righe
\fontdimen2\font=0.5em % imposta spazio tra le parole

\pagenumbering{gobble} % suppress page number

\begin{document}
{\centering \LARGE \bfseries Progetto Tec-Web \\}
\bigskip\bigskip \noindent
\begin{center}\large
	\textbf{\sffamily Studente}: Roberto Ingenito \qquad \textbf{\sffamily Matricola}: N86004077\bigskip\\
	\textbf{\sffamily Traccia}: HiveMind
\end{center}

\subsection*{Tecnologie utilizzate: back-end}
\begin{itemize}
	\item \textbf{\sffamily Express}: framework web per Node.js scelto per costruire le API le quali seguono i principi RESTful.
	\item \textbf{\sffamily Postgres}: database utilizzato
	\item \textbf{\sffamily Sequelize}: ORM che semplifica l'interazione con il database
	\item \textbf{\sffamily bcrypt}: utilizzato per criptare le password degli utenti attraverso l'hashing.
	\item \textbf{\sffamily JWT}: implementato per gestire l'autenticazione e l'autorizzazione degli utenti
\end{itemize}

\subsection*{Tecnologie utilizzate: front-end}
\begin{itemize}
	\item \textbf{\sffamily Angular}: framework principale per lo sviluppo dell'applicazione
	\item \textbf{\sffamily ej2-angular-richtexteditor}: pacchetto di \textit{Syncfusion} per integrare un editor di testo formattato
\end{itemize}

\end{document}


